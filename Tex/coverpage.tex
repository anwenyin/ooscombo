%\newtheorem{theorem}{Theorem}
%\newtheorem{lemma}[theorem]{Lemma}
%\newtheorem{proposition}[theorem]{Proposition}
%\newtheorem{corollary}[theorem]{Corollary}
%\newtheorem{Assumption}{Assumption}

\title{Out-of-Sample Forecast Model Averaging with Parameter Instability}
\author{Anwen Yin\thanks{The author is grateful to Richard Ashley, Joydeep Bhattacharya, Helle Bunzel, Gray Calhoun, David Frankel, Ramazan Gencay, Jonathan McFadden, Jarad Niemi, Dan Nordman, T.J. Rakitan, David Rapach, John Schroeter, Yohei Yamamoto, Guofu Zhou and econometrics session participants in the 2014 Missouri Economics Conference, 2014 Midwest Econometric Group meeting and 31st Canadian Econometric Study Group meeting for helpful comments and suggestions. All errors are mine. Contact: 165 Heady Hall, Department of Economics, Iowa State University, Ames, Iowa, 50011. Telephone: (515) 294-3512. Email: \href{mailto:anwen@iastate.edu}{anwen@iastate.edu}} \\ Iowa State University\\Ames, Iowa}
\date{\today}

\maketitle

\begin{abstract}
  \noindent This paper extends Hansen's (2009) optimal weights combining the stable model and the structural break model to the out-of-sample (\textbf{OOS}) forecast setting. We propose optimal weights based on the leave-one-out cross-validation criterion (\textbf{CV}) to replace Hansen's weights, as CV can be applied more generally compared with Mallows' criterion (\textbf{Cp}), especially in that it is robust to heteroscedasticity which is relevant to many economic time series. These weights are optimal in the sense of minimizing the sample cross-validation criterion under this setting. We provide Monte Carlo evidence showing that CV weights outperform several other methods (i.e. Mallows' weights, equal weights, and Schwarz-Bayesian weights) in several designs. Last, we apply our method to forecasting the U.S. and Taiwan quarterly GDP growth rates out-of-sample and demonstrate the improved performance of our method.\\

  \noindent Keywords: \emph{Cross-Validation, Conditional Heteroscedasticity, Structural Break, Out-of-Sample, Forecast Evaluation, Forecast Combination}\\

  \noindent \textsc{JEL} Classification: C22, C52, C53
\end{abstract}
\newpage
\doublespacing
