\newtheorem{theorem}{Theorem}[section]
\newtheorem{lemma}[theorem]{Lemma}
\newtheorem{proposition}[theorem]{Proposition}
\newtheorem{corollary}[theorem]{Corollary}
\newtheorem{Assumption}{Assumption}

\title{Out-of-Sample Forecast Model Averaging with Parameter Instability}
\author{Anwen Yin\thanks{The author thanks Joydeep Bhattacharya, Helle Bunzel, Gray Calhoun, David Frankel, Jonathan McFadden, Jarad Niemi, Dan Nordman, T.J. Rakitan, David Rapach and Guofu Zhou for helpful comments and suggestions. All errors are mine. Contact: 173 Heady Hall, Department of Economics, Iowa State University, Ames, Iowa, 50011. Telephone: (515) 294-2469. Email: \href{mailto:anwen@iastate.edu}{anwen@iastate.edu}} \\ Iowa State University\\Ames, Iowa}
\date{\today}

\maketitle

\begin{abstract}
  \noindent This paper examines the problem of how to forecast a time series variable of interest when there is uncertainty regarding parameter instability. Specifically, if the uncertainty is strong and we decide to combine forecasts from different models, what is the optimal rule to assigning weights by some information criterion? Built upon Hansen's Mallows' model averaging method, we propose using the cross-validation (CV) criterion to combine forecasting models. We show that CV model averaging is approximately equivalent to Hansen's Mallows' model averaging in the absence of conditional heteroscedasticity. Then we derive the cross-validation model averaging weights by allowing for conditional heteroscedasticity. Through two simulation designs and an empirical example of forecasting U.S. quarterly GDP growth rate out-of-sample, we show that CV performs better than Mallows' model averaging, Schwarz-Bayesian model averaging and equal weight combination in terms of better root mean-squared forecast error.\\

  \noindent Keywords: \emph{Cross-Validation, Conditional Heteroscedasticity, Structural Break, Out-of-Sample, Forecast Evaluation, Forecast Combination}\\

  \noindent \textsc{JEL} Classification: C22, C52
\end{abstract}
\newpage
\doublespacing
