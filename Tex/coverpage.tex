\newtheorem{theorem}{Theorem}[section]
\newtheorem{lemma}[theorem]{Lemma}
\newtheorem{proposition}[theorem]{Proposition}
\newtheorem{corollary}[theorem]{Corollary}
\newtheorem{Assumption}{Assumption}

\title{Out-of-Sample Forecast Model Averaging with Parameter Instability}
\author{Anwen Yin\thanks{The author thanks Joydeep Bhattacharya, Helle Bunzel, Gray Calhoun, Sylvain Chassang, David Frankel, Jonathan McFadden, Jarad Niemi, Dan Nordman, T.J. Rakitan, David Rapach and Guofu Zhou for helpful comments and suggestions. All errors are mine. Contact: 173 Heady Hall, Department of Economics, Iowa State University, Ames, Iowa, 50011. Telephone: (515) 294-2469. Email: \href{mailto:anwen@iastate.edu}{anwen@iastate.edu}} \\ Iowa State University\\Ames, Iowa}
\date{\today}

\maketitle

\begin{abstract}
  \noindent This paper examines the problem of how to forecast a time series variable of interest when there is uncertainty regarding parameter instability in the conditional mean. Specifically, if the uncertainty is strong and we decide to combine forecasting models, what is the optimal rule to assigning weights by some information criterion? Built upon Hansen's Mallows' model averaging method, we propose using the cross-validation (CV) criterion to combine forecasting models. First, we show that CV model averaging is approximately equivalent to Hansen's Mallows' model averaging in the absence of conditional heteroscedasticity. Then we derive the cross-validation model averaging weights by allowing for conditional heteroscedasticity and show that they are the correct weights minimizing the population mean squared forecast error in the presence of conditional heteroscedasticity. We provide Monte Carlo evidence that our method performs better than several popular alternatives (i.e. Mallows' weights, equal weights, and Schwarz-Bayesian weights). Last, we apply our method to forecasting the U.S. and Taiwan quarterly GDP growth rates out-of-sample and show the better performance of our method.\\

  \noindent Keywords: \emph{Time Series, Cross-Validation, Conditional Heteroscedasticity, Structural Break, Out-of-Sample, Macroeconometrics, Forecast Evaluation, Forecast Combination}\\

  \noindent \textsc{JEL} Classification: C22, C52, C53
\end{abstract}
\newpage
\doublespacing