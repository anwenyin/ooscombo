\section{Empirical Application}
\subsection{Out-of-sample Forecast under Conditional Homoskedasticity}
Here we apply our theoretical results to forecasting excess US stock returns based on Goyal and Welch's \cite{goyal_welch_RFS2008} monthly dataset \footnote{Data can be obtained at \url{http://www.hec.unil.ch/agoyal/}.}. Yin \cite{yin2012} shows that there is detected structural breaks in the simple univariate forecasting model with stock market variance but the evidence supporting breaks is not strong: the BIC is almost the same between the break model and the stable model, so there is no clear cut on which one we should choose to forecast. Model selection may not be a good option in this case, so we use the model averaging method studied in this paper to forecast out-of-sample and compare its results with other popular forecast combination methods.

Our monthly data runs from September 1974 to December 2012. We reserve the last 36 observations as the evaluation sample while the rest as training sample. The estimation is based on fixed window, though including more window schemes, namely, recursive window and rolling window, is desirable. To calculate optimal CV weights, for the penalty term, we choose the value corresponding to the one with two regressors and with trimming parameter of 0.15 in Hansen \cite{hansen2009averaging} \footnote{MATLAB code to compute the penalty coefficient at different combinations of number of regressors and trimming parameters can be obtained from the author upon request.}.We compute the out--of--sample MSFE for CV averaging, post-break window, equal weights, Bayesian averaging and GR combination and the results are shown in table 1.

From table 1, we can see that the CV model averaging performs significantly better than equal weights, Bayesian model averaging and Granger--Ramanathan (\textbf{GR}) combination in this empirical study. Granger--Ramanathan combination performs the worst among all five methods confirming the prevailing forecast combination puzzle which says that equal weights scheme dominates the GR combination. One fact worth pointing out is that the post-break window forecast performs equally well as CV averaging in this case. 