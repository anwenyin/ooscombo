\section{Related Literature}
This paper closely follows the literature related to information criterion-based model selection and averaging, structural breaks testing, heteroscedasticity and autocorrelation consistent/robust (\textbf{HAC/HAR}) covariance matrix estimation, and out-of-sample forecast comparison and forecast evaluation.

Recently, Hansen has published a series of papers \cite{hansen_EMETRICA2007} \cite{hansen_JE2008} \cite{hansen2009averaging} \cite{hansen2011jackknife} which help develop relevant econometric theory for the use of model averaging under various situations, and has pushed the forecast combination theory to a new level. He establishes that under the assumption of conditional homoskedasticity and the restriction of weight discretization, model averaging estimators based on Mallows' criterion are asymptotically optimal in the sense of minimizing mean squared error while controlling omitted variable bias. The reason for using Mallows' criterion is because it is an asymptotically unbiased estimator of the in-sample MSE or one-step ahead out-of-sample MSFE compared with other criteria, such as Akaike information criterion (\textbf{AIC}) or Schwarz-Bayesian information criterion (\textbf{SIC}). Hansen later extends his Mallows' model averaging theory to forecast combination and compares its performance with other related combination methods based on simulated data \cite{hansen_JE2008}. He shows that Mallows' criterion is an approximately unbiased estimator of MSFE even for a stationary time series, but the optimality results do not apply. In order for the optimality to hold, we need the data to be independent and identically distributed. This restriction has made the optimality property less relevant for many empirical applications where the data under study is time series, for example, GDP growth rate, unemployment rate and inflation rate. Even more stringently, Hansen imposes the restriction that the models under consideration are strictly nested\footnote{Hansen considers a sequence of nested MA models.} in order to ensure optimality. In another paper coauthored with Racine \cite{hansen2011jackknife}, Hansen relaxes the assumption of conditional homoscedasticity and nested linear models to show model averaging optimality by replacing the Mallows' criterion with the cross-validation criterion, but the optimality is still restricted to random samples and does not allow arrival of new models for consideration. Why cross-validation? Comparing Mallows' criterion with CV, Andrews \cite{andrews_JE1991} demonstrates that Mallows' criterion is no longer optimal in model selection if allowing for conditional heteroscedasticity, and CV is the only feasible criterion among popular candidates that are asymptotically optimal under general conditions. Liu and Okui \cite{liu_okui2012} propose a heteroscedasticity-robust Mallows' criterion which generalizes Hansen's least squares model averaging optimality results by allowing for conditionally heteroscedastic errors. One major drawback of the model averaging methods discussed above is that none of the optimality results applies to the setting where structural breaks are possible or nonlinear models are present.

Model averaging under parameter instability relies heavily on the theory of break detection, break dates estimation and inference. Historically, applied econometricians rely on the Chow test to deal with structural break, but the use of Chow's test assumes that the researcher knows the exact date of the structural break, if it indeed happens. This seems quite unrealistic and requires that econometricians visually examine the time series data to search for a possible break point, an arbitrary process. In a seminal paper, Andrews \cite{andrews93} has shown the non-standard asymptotic distribution of a class of Sup-type test statistic for detecting breaks and conducting inference when the break point is unknown. This results in a subsequent series of articles related to break detection and optimal testing, notably Andrews \cite{andrews_ploberger94} \cite{andrews2003}, Hansen \cite{hansen_JE2000}, Bai \cite{bai_ET1997} \cite{bai_JE1999}, Bai and Perron \cite{bai_perron98}, Elliott and Muller \cite{elliott_muller_RES2006}, Rossi \cite{rossi_ET2005} and Bunzel and Iglesias \cite{bunzel_iglesias}. Bai and Perron \cite{bai_perron98} generalize Andrews' test by introducing methods to detect and date multiple change points in a partial structural break linear model, as well as develop relevant asymptotic theory for conducting inference. Bai and Perron's computational procedure for detecting breaks is adopted in many empirical works related to macroeconomic and financial time series since it is reasonable to think that there could be multiple structural breaks, for example, the U.S. equity markets have experienced institutional change and several financial crises since the early twentieth century.

From the perspective of a forecaster, testing for structural breaks is not the end. How to better predict the future and evaluate forecasts is of great importance to econometricians working on economic forecasting. There are several papers related to forecasting with breaks. See Pesaran and Timmermann \cite{pesaran_timmermann_JE2007}, Pesaran, Pick and Pranovich \cite{pesaran_pick_pranovich_2011}. The focus of this string of literature is on the forecasting window choice. Specifically, what portion of the data to use. The selection of forecasting window involves a bias-variance trade-off. Including more data before the estimated break date may help reduce the mean squared forecast error, but doing so could result in more bias in the parameter estimation.

As an alternative to the forecasting model averaging method studied in this paper when parameter instability is possible, several researchers have proposed various in--sample and out--of--sample tests to select a forecasting model which is robust to structural breaks. See Giacomini and Rossi \cite{giacomini_rossi_2008} \cite{giacomini_rossi_2010}, Bunzel and Calhoun \cite{bunzel_calhoun_2012} and Inoue and Kilian \cite{inoue_kilian_ER2004}.

Since allowing for heteroscedasticity requires the use of HAC or HAR estimators of the covariance matrix, various methods regarding the choice of kernel and bandwidth for estimating covariance matrix can be considered. See Newey and West \cite{newey_west_EMETRICA1987}, Andrews \cite{andrews91}, Kiefer, Vogolsang and Bunzel \cite{kvb2000} and Sun \cite{sunyixiao_2010}. 