\section{Conclusion}
We are interested in answering a basic question of how to forecast a time series variable of interest when there is uncertainly about parameter instability. Specifically, which model should be used for forecasting: the break model or the stable one? If uncertainty is strong and we decide to combine forecasts from two models, what is the optimal rule in terms of some information criterion about assigning weights? Built upon Hansen's Mallows' model averaging method, we propose using the cross-validation criterion to combine forecasting models. In the literature of model selection, CV is shown to be more robust to heteroscedasticity than other information criteria, such as, AIC, BIC and Mallows'. Without assuming conditional heteroscedasticity, we show that CV model averaging is approximately equivalent to Hansen's Mallows' model averaging. But in many empirical applications related to macroeconomic time series or financial time series, researchers can usually not avoid explicitly dealing with heteroscedasticity for analysis and forecast. This motivates our generalization of the cross-validation model averaging method by allowing for conditional heteroscedasticity.

Researchers have found that in many applications, equally weighted forecasts perform better than other complex combination methods. This forecast combination puzzle has cast double on the use of complicated model averaging methods. Both CV and Cp weights are easy to compute and do not rely on direct weight estimation as in the Granger-Ramanathan forecast combination. This feature should be appealing to practitioners and professional forecasters because simplicity can help reduce the excess noise introduced by using complex weighting methods. This may help explain why our methods forecast better than equal weight in both controlled simulation and in forecasting U.S. quarterly GDP growth rate out-of-sample.

