\section{Simulation Results}
Here we are going to evaluate the forecast performance of CV model averaging through controlled numerical simulation. Specifically, we are going to use three different designs of the true data generating process: (i). an AR(2) process plus five exogenous predictors with ARCH(1) errors,
\begin{subequations}
            \begin{align}
                y_{t} & = \mu + \rho_{1}y_{t-1} + \rho_{2}y_{t-2} + \sum_{i=1}^{5}\theta_{i}x_{i} + e_{t} \\
                e_{t} & = v_{t}\sqrt{h_{t}}\\
                h_{t} & = \alpha_{0} + \alpha_{1}e_{t-1}^2
            \end{align}
\end{subequations}
(ii). an AR(2) process plus two exogenous predictors with heteroscedastic errors drawing from this distribution $N(0,y_{t-1}^2)$.
\begin{equation}
                y_{t} = \mu + \rho_{1}y_{t-1} + \rho_{2}y_{t-2} + \sum_{i=1}^{2}\theta_{i}x_{i} + e_{t}
\end{equation}
(iii). an AR(2) process with a break in the variance of the error term. We consider this design to study the forecasting performance of CV model averaging in the Great Moderation type environment. The break date of the error term variance is not identical to that of the conditional mean\footnote{In the simulation we set the break fraction of the error term variance at $0.5$ relative to the training sample. But the break fraction for the conditional mean is set at $0.3$ relative to the training sample.}. We allow for this break date difference hoping to better approximate the environment forecasters face in practice. Mathematically, the data generating process for design (iii) is the following:
\begin{equation}
            y_{t} = \mu + \rho_{1}y_{t-1} + \rho_{2}y_{t-2} + e_{t}
\end{equation}
where
\[ e_{t} \sim \left\{
  \begin{array}{l l}
    N(0,\sigma^2) & \quad t \in [1,\tau_{v}]\\
    N(0,\frac{1}{4}\sigma^2) & \quad t \in [\tau_{v}+1, R]
\end{array} \right.\]
In all three designs there is a one-time structural break in all coefficients of the conditional mean happening at the $30\%th$ observation of the training sample $R$, that is, $\tau = 0.3$. We let the structural break of the parameters take the multiplicative form, that is, if the pre-break coefficient is $\beta$, the post-break parameter becomes $\delta\beta$, where $\delta$ controls the break size. For the ARCH process, $v_{t}$s are drawn independently and identically from the standard normal distribution. Other predictors are drawn i.i.d as the following: $x_{1} \sim \mathrm{N}(0,4)$, $x_{2} \sim \mathrm{U}[-2,2]$, $x_{3} \sim \mathrm{N}(0,16)$, $x_{4} \sim \mathrm{t}(5)$ and $x_{5} \sim \mathrm{Binomial}(1,0.02)$. The parameter values for all data generating processes are $\mu = 2, \rho_{1} = 0.4, \rho_{2} = 0.2$, $\theta_{1} = 0.8,\theta_{2} = -0.4,\theta_{3} = 2,\theta_{4} = -3.5,\theta_{5} = 10$, $\alpha_{0} = 1, \alpha_{1} = 0.4$. These values are chosen to satisfy the stationarity and ARCH error regularity assumptions. Note that in our simulation, the post-break coefficient values become smaller than their pre-break counterparts. This choice of break direction provides us with more freedom in controlling the break size. For example, if the true data generating process is an intercept-free AR(1) model with pre-break parameter value $0.9$, to ensure regime-wise stationarity, $\delta$ should not take values greater than $1.1$ if we prefer larger post-break parameter value\footnote{Bai and Perron \cite{bai_perron98} assume that the break size is large enough in order to be identified and estimated. Though we have not found any leading metric measuring the break size, break size of $1.1$ as in the example is not large enough to identify especially when the data is highly volatile as shown in our simulation work.}.

To make our simulation design close to relevant empirical studies and to capture the difficulty many researchers face in finding the best approximating models, the model used to forecast differs from the true data generating process\footnote{The difference of the AR order between the DGP and the forecasting model captures the fact that in practice, it is hard to fully capture the dynamics by selecting the true order. By the principle of parsimony, researchers or practitioners tend to select a model of small order.}: in case (i) the model to forecast is based those five exogenous predictors in the DGP, $y_{t} = \mu + \sum_{i=1}^{5}\theta_{i}x_{i} + e_{t}$; in case (ii), again the model to forecast does not involve the AR component, $y_{t} = \mu + \sum_{i=1}^{2}\theta_{i}x_{i} + e_{t}$; in case (iii), the model to forecast is the AR(1) with intercept, $y_{t} = \mu + \rho_{1}y_{t-1} + e_{t}$.

For each case, we evaluate the out-of-sample (OOS) forecasting performance by comparing the root mean squared forecast error divided by that of the equal weights method. Recursive window is used to generate OOS forecasts as it mimics the practice that forecasters update their forecast when new data become available. Out-of-sample forecast is constructed by the following steps. First, we split the time series sample into two parts: the prediction or training sample of size $R$ and the evaluation sample of size $P$. Under the recursive window, at each point in time, the estimated parameter is updated by adding one more observation starting with sample size $R$. For example, $\beta_{t} = (\sum_{s=1}^{t-1} x_{s} x_{s}')^{-1}\sum_{s=1}^{t-1}x_{s}y_{s+1},\beta_{t+1} = (\sum_{s=1}^{t} x_{s} x_{s}')^{-1}\sum_{s=1}^{t}x_{s}y_{s+1}$. By this procedure, we estimate parameters recursively and then generate a sequence of forecasts of size $P$ based on these estimated parameters. We can compare this sequence of forecasts with those reserved data in the evaluation sample, and evaluate our forecasts according to some loss function, for example, RMSFE or MSFE. See Calhoun \cite{calhoun_bootstrap_2013} \cite{calhoun_overfit_2014}, McCracken \cite{mccracken_JE2000} \cite{mccracken_JE2007}, Rossi \cite{rossi_HANDBOOK2013}, Clark and McCracken \cite{clark_mccracken_JE2001} \cite{clark_mccracken_JE2005} \cite{clark_mccracken_HANDBOOK2013}, Clark and West \cite{clark_west_JE2007} and West \cite{west_HANDBOOK2006} for more details on out-of-sample forecasting.

The total sample size, $T$, is $200$\footnote{This sample size is chosen to be relevant to most macroeconomic time series.}. In our pseudo one-step ahead out-of-sample forecasting simulation, we reserve the first $170$ or $150$ ($R = 170$ or $R = 150$) observations as the training sample and the rest as the prediction sample ($P = 30$ or $P = 50$). For the break model, we use the post-break window method to forecast out-of-sample. Other techniques, such as the optimal window method proposed by Pesaran and Timmermann \cite{pesaran_timmermann_JE2007} or the robust weight method proposed by Pesaran, Pick and Pranovich \cite{pesaran_pick_pranovich_2011} could also be considered. For simplicity, we only apply the post-break window method in this paper \footnote{Currently, researchers are sill working on developing theory and methods related to forecasting with breaks, and we are not aware of any dominant method that performs well in most situations faced by practitioners. The simulation conducted by Pesaran and Timmermann suggests that there is little gain from complicated methods. The simple rule, to forecast using the data after the detected break, seems to work as well as anything else.}.

In each case, to evaluate and compare performance, we produce forecasts using six methods \footnote{Methods such as Bates-Granger combination, Granger-Ramanathan combination and common factor combination are not considered in our simulation. In a related paper, Clark and McCracken \cite{clark_mccracken_averagingVAR_2008} conclude that ``\emph{...it is clear that the simplest forms of model averaging---such as those that use equal weights across all models---consistently perform among the best methods...forecasts based on OLS-type combination and factor-based combination rank among the worst}". So we only compare our method with either closely related or empirically proven effective methods.}:
\begin{inparaenum}[(i)]
\item Mallows' model averaging (\textbf{Cp});
\item CV model averaging (\textbf{CV});
\item Bayesian model averaging\footnote{We call this method ``Bayesian'' not in a strict sense: the Bayesian weight for each model is calculated based on the value of the Schwarz-Bayesian information criterion, i.e. the weight for the beak model is $w_{b} = \exp{(SIC^{b})}/(\exp{(SIC^{b})} + \exp{(SIC^{s})})$} (\textbf{SIC});
\item stable model (\textbf{Stable});
\item break model (\textbf{Break}); and
\item equal weights\footnote{Each model receives weight of $0.5$.} (\textbf{Equal}).
\end{inparaenum}
We evaluate their forecast performance by root mean-square forecast error (\textbf{RMSFE}). For ease of comparison, we pick the equal weight method as the benchmark\footnote{The reason to pick equal weights as the benchmark is because of the aforementioned forecast combination puzzle: equally weighted forecasts tend to perform better than other complicated methods in many applications. The puzzle is generally discussed in the forecasting literature without allowing breaks. In this study we try to examine whether it dominates our method when facing structural breaks.} and compute the relative performance (\textbf{Ratio}) for each method, for example, RMSFE\textsuperscript{CV}/RMSFE\textsuperscript{Equal}. If the ratio is less than one, it indicates better performance than equal weights. The smaller the ratio is, the better the forecasting performance is for given sample split.
\subsection{Design I}
\begin{table}
    \caption{Monte Carlo Simulation: Design I} \label{ntb:1}
    \centering
    \begin{adjustbox}{width=\textwidth,totalheight=\textheight,keepaspectratio}
    \begin{threeparttable}
    \begin{tabular}{ccccccccccc}
    \toprule
     & \multicolumn{5}{c}{$P = 30$} & \multicolumn{5}{c}{$P = 50$} \\
    \cmidrule(r){2-6}
    \cmidrule(r){7-11} \\
      Break Size   & Cp     &   CV   & SIC    & Stable & Break  & Cp     & CV     & SIC    & Stable & Break \\
             $100$ & 0.6312 &0.6298  &1.2987  & 1.6557 &0.6297  & 0.6599 &0.6585  &1.2849  & 1.6220 &0.6584 \\
             $10$  & 0.6644 &0.6627  &1.2563  & 1.6148 &0.6627  & 0.6871 &0.6854  &1.2473  & 1.5874 &0.6853 \\
             $5$   & 0.7085 &0.7066  &1.2063  & 1.5605 &0.7065  & 0.7289 &0.7271  &1.2005  & 1.5335 &0.7270 \\
             $3$   & 0.7658 &0.7636  &1.1517  & 1.4782 &0.7636  & 0.7869 &0.7850  &1.1454  & 1.4489 &0.7850 \\
             $2$   & 0.8330 &0.8308  &1.0974  & 1.3734 &0.8308  & 0.8500 &0.8483  &1.0925  & 1.3471 &0.8483 \\
    \bottomrule
    \end{tabular}
    \begin{tablenotes}[para, flushleft] \footnotesize
    Notes: The DGP is $y_{t} = \mu + \rho_{1}y_{t-1} + \rho_{2}y_{t-2} + \sum_{i=1}^{5}\theta_{i}x_{i} + e_{t}, e_{t} = v_{t}\sqrt{h_{t}}, h_{t} = \alpha_{0} + \alpha_{1}e_{t-1}^2$ and the forecasting model is $y_{t} = \mu + \sum_{i=1}^{5}\theta_{i}x_{i} + e_{t}$. $\mathrm{P}$ is the evaluation sample size, total sample size is $200$, break fraction relative to the training sample is $\tau = 0.3$, OOS forecasts are generated by the recursive window, 5000 times replication. Equal weight is chosen as the benchmark and the numbers in the table represent the RMSFE ratio between each individual method and equal weight. Cp: Mallows' weights. CV: cross-validation weights. SIC: Schwarz-Bayesian weights. Stable: model without structural break. Break: model with a full structural break.
    \end{tablenotes}
    \end{threeparttable}
    \end{adjustbox}
\end{table}
Simulation results for the ARCH\footnote{Our results also hold in the GARCH error case.} error design are presented in table \ref{ntb:1}. We can see from the table that CV forecasts better than Cp across all considered break sizes and prediction sample sizes. Both of CV and Cp's relative RMSFE decreases monotonically as the break size increases, but CV decreases at a slightly faster speed. Bayesian weighting does slightly worse than the equal weight method, but its performance deteriorates when the break size becomes large as it fails to capture the fact that the evidence supporting break is becoming stronger. It should not be a surprise that the break model does well since structural break indeed happens in the DGP, but it performs slightly worse than CV because it does not take the volatility into account.

Our results indicate that when there is ARCH type conditional heteroscedasticity in the data and when the break is not strictly dominated by the volatility, CV forecasts better than Mallows' model averaging. Additionally, CV forecasts better than equal weight so the forecast combination puzzle does not apply in this design. Bayesian model averaging is approximately equivalent to equal weight, but it is less sensitive to the size of break. Compared with CV, Bayesian weighting method does not put more weight on the proper model even the break size increases significantly.

It is worth mentioning that in our design, the post-break coefficients become smaller than their pre-break counterparts by various degrees. This procedure is adopted to ensure that piece-wise stationarity is maintained under structural break. \footnote{The simulation results stay the same if we reverse the break size direction by starting with smaller pre-break parameter values.}
\subsection{Design II}
\begin{table}
    \caption{Monte Carlo Simulation: Design II} \label{ntb:2}
    \centering
    \begin{adjustbox}{width=\textwidth,totalheight=\textheight,keepaspectratio}
    \begin{threeparttable}
    \begin{tabular}{ccccccccccc}
    \toprule
     & \multicolumn{5}{c}{$P = 30$} & \multicolumn{5}{c}{$P = 50$} \\
    \cmidrule(r){2-6}
    \cmidrule(r){7-11} \\
    Break Size      & Cp     & CV     & SIC    & Stable & Break  & Cp     & CV     & SIC    & Stable & Break  \\
    $100$           & 0.4610 &0.2586  &1.0717  & 1.9477 &0.2586  & 0.5706 &0.3415  &1.0649  & 1.8951 &0.3415 \\
    $10$            & 0.7007 &0.5681  &1.0393  & 1.6830 &0.5683  & 0.6945 &0.5419  &1.0392  & 1.6930 &0.5421 \\
    $5$             & 0.8422 &0.7700  &1.0194  & 1.4212 &0.7701  & 0.8699 &0.7946  &1.0191  & 1.3916 &0.7948 \\
    $3$             & 0.8978 &0.8541  &1.0111  & 1.2809 &0.8543  & 0.9135 &0.8800  &1.0126  & 1.2551 &0.8803 \\
    $2$             & 0.9188 &0.8778  &1.0082  & 1.2352 &0.8781  & 0.9417 &0.9320  &1.0074  & 1.1578 &0.9323 \\
    \bottomrule
    \end{tabular}
    \begin{tablenotes}[para, flushleft] \footnotesize
    Notes: The DGP is $y_{t} = \mu + \rho_{1}y_{t-1} + \rho_{2}y_{t-2} + \sum_{i=1}^{2}\theta_{i}x_{i} + e_{t}, e_{t} \sim \mathrm{N}(0, y_{t-1}^{2})$ and the forecasting model is $y_{t} = \mu + \sum_{i=1}^{2}\theta_{i}x_{i} + e_{t}$. $\mathrm{P}$ is the evaluation sample size, total sample size is $200$, break fraction relative to the training sample is $\tau = 0.3$, OOS forecasts are generated by the recursive window, 5000 times replication. Equal weight is chosen as the benchmark and the numbers in the table represent the RMSFE ratio between each individual method and equal weight. Cp: Mallows' weights. CV: cross-validation weights. SIC: Schwarz-Bayesian weights. Stable: model without structural break. Break: model with a full structural break.
    \end{tablenotes}
    \end{threeparttable}
    \end{adjustbox}
\end{table}
Simulation results for the second design are shown in table \ref{ntb:2}. In this case we can see from the table that CV forecasts much better than Cp across all break sizes and prediction sample sizes. Both of their relative RMSFE decreases monotonically as the break size increases, but in this design the RMSFE of CV decreases at a much faster speed. Bayesian weighting does almost the same as equal weight, but its performance deteriorates when the break size becomes large as shown in the previous design. The pattern of our results holds for both cases of the out-of-sample size.

Our results indicate that when there is this ``wild'' type heteroscedasticity in the data as modeled in the DGP and when the break is not strictly dominated by the volatility, CV forecasts significantly better than Mallows' model averaging, especially when the break size is large. Additionally, CV forecasts better than equal weight so the forecast combination puzzle does not apply in this design. Bayesian model averaging is approximately equivalent to equal weight. Again, compared with CV, Bayesian weighting method does not put more weight on the proper model when the break size increases.
\subsection{Design III}
\begin{table}
    \caption{Monte Carlo Simulation: Design III} \label{ntb:4}
    \centering
    \begin{adjustbox}{width=\textwidth,totalheight=\textheight,keepaspectratio}
    \begin{threeparttable}
    \begin{tabular}{ccccccccccc}
    \toprule
     & \multicolumn{5}{c}{$P = 30$} & \multicolumn{5}{c}{$P = 50$} \\
    \cmidrule(r){2-6}
    \cmidrule(r){7-11} \\
    Break Size      & Cp     & CV     & SIC    & Stable & Break  & Cp     & CV     & SIC    & Stable & Break  \\
    $100$           & 0.9810 &0.9759  &1.0011  & 1.0839 &0.9760  & 0.9825 &0.9769  &1.0011  & 1.0789 &0.9770 \\
    $10$            & 0.9860 &0.9789  &1.0006  & 1.0716 &0.9790  & 0.9880 &0.9822  &1.0006  & 1.0656 &0.9823 \\
    $5$             & 0.9919 &0.9850  &1.0003  & 1.0583 &0.9852  & 0.9933 &0.9868  &1.0003  & 1.0534 &0.9870 \\
    $3$             & 0.9977 &0.9903  &1.0000  & 1.0455 &0.9906  & 0.9975 &0.9905  &1.0001  & 1.0428 &0.9908 \\
    $2$             & 1.0009 &0.9940  &0.9999  & 1.0347 &0.9944  & 1.0013 &0.9952  &0.9999  & 1.0316 &0.9958 \\
    \bottomrule
    \end{tabular}
    \begin{tablenotes}[para, flushleft] \footnotesize
    Notes: The DGP is $y_{t} = \mu + \rho_{1}y_{t-1} + \rho_{2}y_{t-2} + e_{t}$, $e_{t} \sim N(0,\sigma^2) \quad t \in [1,\tau_{v}]$ and $e_{t} \sim N(0,\frac{1}{4}\sigma^2) \quad t \in [\tau_{v}+1, R]$, $\tau_{v} = 0.5R$, the forecasting model is $y_{t} = \mu + \rho_{1}y_{t-1} + e_{t}$. $\mathrm{P}$ is the evaluation sample size, total sample size is $200$, break fraction relative to the training sample is $\tau = 0.3$, OOS forecasts are generated by the recursive window, 5000 times replication. Equal weight is chosen as the benchmark and the numbers in the table represent the RMSFE ratio between each individual method and equal weight. Cp: Mallows' weights. CV: cross-validation weights. SIC: Schwarz-Bayesian weights. Stable: model without structural break. Break: model with a full structural break.
    \end{tablenotes}
    \end{threeparttable}
    \end{adjustbox}
\end{table}
Simulation results for this Great Moderation type design are shown in table \ref{ntb:4}. The general pattern shown in the previous two designs remains in this case. CV forecasts better than Cp across all listed break sizes and prediction sample sizes. Both of their relative RMSFE decreases monotonically as the break size increases, but the relative RMSFE of CV decreases at a slightly faster speed. Bayesian weighting does almost the same as equal weight, but its performance is less sensitive to the break size in this case. The pattern of our results holds for both cases of the out-of-sample size.

In this design our results indicate that when the data has the Great Moderation appearance as shown in the U.S. GDP growth data, and when the break is not strictly dominated by the volatility, CV forecasts better than Mallows' model averaging. Additionally, the forecast combination puzzle does not apply here. This motivates us to apply our method to forecasting real U.S. GDP growth rate and to comparing its performance with other related methods.
